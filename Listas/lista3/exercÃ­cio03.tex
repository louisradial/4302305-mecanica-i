\section*{Exercício 3}
Sendo \(\vec{r}_M = x\hat{\jmath}\) a posição da partícula de massa \(M\), a posição da partícula de massa \(m\) é dada por
\begin{equation*}
    \vec{r}_m = \vec{r}_M + \ell \left(\cos\varphi \hat\imath + \sin\varphi \hat\jmath\right),
\end{equation*}
onde \(\ell\) é o comprimento do fio inextensível. Assim, temos
\begin{align*}
    \dot{\vec{r}}_m &= \dot{\vec{r}}_M + \ell \dot\varphi(-\sin\varphi\hat\imath + \cos\varphi\hat\jmath)\\
                    &= \left(-\ell\dot\varphi\sin\varphi\right)\hat\imath + \left(\dot{x} + \ell\dot\varphi\cos\varphi\right)\hat\jmath
\end{align*}
como a velocidade da partícula de massa \(m\).
