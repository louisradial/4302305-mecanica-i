\section*{Exercício 1}
Como mostrado na \cref{fig:exercício1}, \(x\) é a distância vertical da polia até a massa \(M\) e \(r\) é distância da polia até a massa \(m\). Então para um fio inextensível, devemos ter \(x + r = \ell,\) onde \(\ell = L - D\) é a constante dada pelo comprimento total \(L\) do fio menos a distância \(D\) entre as polias.

\begin{figure}[h]
   \centering
   \begin{tikzpicture}[thick,scale=0.75, every node/.style={scale=0.6}]
       \usetikzlibrary{calc,patterns}
       % Supporting structure
       \coordinate (P1) at (5,0);
       \coordinate (P2) at (-5,0);
       \fill[pattern=crosshatch dots] (-7,1.61) rectangle (7,1.21);
       \draw(-7,1.21) -- (7,1.21);

       % Cable
       \draw[black,thick] ($(P1) + (0,0.75cm)$) -- ($(P2) + (0,0.75cm)$) node[midway,below]{\(D\)};
       \draw[black, thick] ($(P2) + (-0.75cm, 0)$) -- +(-90:6) coordinate (M) node[midway, left]{\(x\)};
       \draw[black, thick] ($(P1) + (15:0.75cm)$) coordinate (P) -- +(-75:6) coordinate (m) node[midway,right]{\(r\)};

       % Angle
       \draw[black, dashed, thick] (P) -- +(-90:6);
       \draw[dashed] (m) arc(-75:-90:6);
        \filldraw [fill=Mauve!5, draw=Mauve!50!black] (P) -- +(-75:3)
            arc [start angle = -75, end angle = -90, radius = 3] node[midway, below, Mauve] {\(\varphi\)};

       % Pulley
       \draw[fill = gray] (P1) circle (0.75cm); % Big circle
       \draw[fill=lightgray] (P1) circle (0.65cm); % Medium circle
       \draw[fill=white] ($(P1) + 0.5*(75:2.5)$) to[rounded corners=0.1cm] ($(P1)+0.5*(0.2,-0.25)$)
       to[rounded corners=0.1cm] ($(P1)+0.5*(-0.2,-0.25)$) -- ($(P1)+0.5*(105:2.5)$) -- cycle;
       \draw[fill=darkgray] (P1) circle (0.06cm); % Axle circle
       % Pulley
       \draw[fill = gray] (P2) circle (0.75cm); % Big circle
       \draw[fill=lightgray] (P2) circle (0.65cm); % Medium circle
       \draw[fill=white] ($(P2) + 0.5*(75:2.5)$) to[rounded corners=0.1cm] ($(P2)+0.5*(0.2,-0.25)$)
       to[rounded corners=0.1cm] ($(P2)+0.5*(-0.2,-0.25)$) -- ($(P2)+0.5*(105:2.5)$) -- cycle;
       \draw[fill=darkgray] (P2) circle (0.06cm); % Axle circle
        \draw[line width=0.6,Mauve!30!black,fill=Mauve!40!black!10,rounded corners=1,top color=Mauve!40!black!20,bottom color=Mauve!40!black!10,shading angle=20] (M) circle(0.4) node{\(M\)};
        \draw[line width=0.6,Pink!30!black,fill=Pink!40!black!10,rounded corners=1,top color=Pink!40!black!20,bottom color=Pink!40!black!10,shading angle=20] (m) circle(0.3) node{\(m\)};
   \end{tikzpicture}
   \caption{Máquina de Atwood}
   \label{fig:exercício1}
\end{figure}

Com isso, as coordenadas generalizadas escolhidas são \(r, \varphi\), de modo que as posições das massas \(M\) e \(m\) são dadas por
\begin{equation*}
    \vetor{r}_M = (\ell - r)\vetor{\hat\imath}-D \vetor{\hat\jmath}\quad\text{e}\quad\vetor{r}_m = r\left(\cos\varphi \vetor{\hat\imath} + \sin\varphi \vetor{\hat\jmath}\right).
\end{equation*}
Assim, a energia cinética do sistema é dada por
\begin{equation*}
    T = \frac12 M\dot{r}^2 + \frac12m \left(\dot{r}^2 + r^2\dot\varphi^2\right).
\end{equation*}
Colocando a referência do potencial gravitacional em \(\ell \vetor{\hat\imath}\), temos
\begin{equation*}
    U = Mgr + mg\left(\ell - r\cos\varphi\right).
\end{equation*}
Dessa forma, a lagrangiana \(\mathcal{L}\) do sistema é dada por
\begin{align*}
    \mathcal{L} &= \frac12 M\dot{r}^2 + \frac12m \left(\dot{r}^2 + r^2\dot\varphi^2\right) - Mgr - mg\left(\ell - r\cos\varphi\right)\\
                &= \frac{M+m}{2}\dot{r}^2 + \frac{m}{2}r^2\dot\varphi^2 - (M - m\cos\varphi)gr - mg\ell.
\end{align*}

Utilizando as equações de Euler-Lagrange, obtemos as equações de movimento do sistema. Temos
\begin{equation*}
    \diff*{{\left[\diffp{\mathcal{L}}{\dot{r}}\right]}}{t} - \diffp{\mathcal{L}}{r} = 0 \implies (M+m)\ddot{r} - mr\dot\varphi^2 - (M - m\cos\varphi)g = 0
\end{equation*}
e
\begin{equation*}
    \diff*{{\left[\diffp{\mathcal{L}}{\dot{\varphi}}\right]}}{t} - \diffp{\mathcal{L}}{\varphi} = 0 \implies mr^2\ddot\varphi + 2mr\dot{r}\dot\varphi + mgr\sin\varphi = 0.
\end{equation*}
Assim, as equações de movimento são dadas por
\begin{equation*}
    \begin{cases}
            (M+m)\ddot{r} - mr\dot\varphi^2 - (M - m\cos\varphi)g = 0\\
            r\ddot\varphi + 2\dot{r}\dot\varphi + g\sin\varphi = 0
    \end{cases}.
\end{equation*}
