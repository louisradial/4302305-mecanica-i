\section*{Exercício 1}
\begin{exercício}{Movimento de uma partícula sujeita a um sistema de vínculos}{exercício 1}
    Considere o movimento de uma partícula em três dimensões que está sujeita aos vínculos
    \begin{enumerate}[label=(\alph*)]
        \item \((x^2 + y^2)\dl{x} + xy\dl{z} = 0\quad\text{e}\quad(x^2+y^2)\dl{y} + yz\dl{z} = 0;\)
        \item \((x^2 + y^2)\dl{x} + xz\dl{z} = 0\quad\text{e}\quad(x^2+y^2)\dl{y} + yz\dl{z} = 0.\)
    \end{enumerate}
    Decida se cada um dos sistemas é holonômico.
\end{exercício}
\begin{proof}[Resolução do item (a)]
    Dividindo os vínculos por \(\dl{t}\) obtemos
    \begin{equation*}
        a_{11}\dot{x} + a_{13} \dot{z} = 0\quad\text{e}\quad a_{22}\dot{y} + a_{23}\dot{z} = 0,
    \end{equation*}
    onde \(a_{11} = a_{22} = x^2+y^2,\) \(a_{13} = xy,\) e \(a_{23} = yz\). Utilizando a lagrangiana \(L = \frac12 m \left(\dot{x}^2 + \dot{y}^2 + \dot{z}^2\right)\), obtemos as equações de movimento
    \begin{equation*}
        \begin{cases}
            m\ddot{x} = \lambda^1a_{11}\\
            m\ddot{y} = \lambda^2a_{22}\\
            m\ddot{z} = \lambda^1a_{13} + \lambda^2a_{23}\\
            a_{11}\dot{x} + a_{13} \dot{z} = 0\\
            a_{22}\dot{y} + a_{23}\dot{z} = 0,
        \end{cases}
    \end{equation*}
    para multiplicadores de Lagrange \(\lambda^1\) e \(\lambda^2\).

    Multiplicando as duas últimas equações por \(\lambda^1\) e \(\lambda^2\) e somando-as, obtemos
    \begin{equation*}
        \lambda^1 a_{11} \dot{x} + \lambda^2 a_{22} \dot{y} + \left(\lambda^1 a_{13} + \lambda^2 a_{23}\right)\dot{z} = 0.
    \end{equation*}
    Substituindo as três primeiras equações temos
    \begin{equation*}
        m\left(\dot{x}\ddot{x} + \dot{y}\ddot{y} + \dot{z}\ddot{z}\right) = 0 \implies \diff*{\left(\frac12mv^2\right)}{t} = 0.
    \end{equation*}
    Isto é, a energia cinética é uma integral de movimento deste sistema, logo as forças não realizam trabalho. Dessa forma, as forças são de vínculo, portanto o sistema é holonômico.
\end{proof}
\begin{proof}[Resolução do item (b)]
    Integremos diretamente o sistema, notando que podemos isolar \(z \dl{z}\) em ambas equações. Temos então a equação diferencial ordinária a variáveis separáveis
    \begin{equation*}
        \frac{\dl{x}}{x} = \frac{\dl{y}}{y},
    \end{equation*}
    cuja solução é \(y = kx\), para alguma constante de integração \(k \in \mathbb{R}\). Substituindo na primeira equação de vínculo, obtemos
    \begin{equation*}
        z \dl{z} = - (1 + k^2)x \dl{x} \implies x^2(1 + k^2) +  z^2 = \ell^2 \implies x^2 + y^2 + z^2 = \ell^2,
    \end{equation*}
    para uma constante de integração \(\ell \in \mathbb{R}\). Deste modo, concluímos que este vínculo é equivalente ao vínculo holonômico da interseção do plano \(y - kx = 0\) com a esfera de raio \(\ell\) centrada na origem \(x^2 + y^2 + z^2 = \ell^2\), portanto o sistema é holonômico.
\end{proof}
