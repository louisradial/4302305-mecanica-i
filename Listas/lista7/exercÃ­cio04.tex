\section*{Exercício 4}
\begin{exercício}{}{exercício04}
    Um anel fino de massa \(m\) e raio \(R\) oscila num plano vertical em torno do ponto fixo \(O\), como mostrado na \cref{fig:exercício04}. Uma conta de massa \(m\) move-se sem atrito ao redor do anel.
    \begin{center}
        \includegraphics[width=0.4\linewidth]{exercício04.png}
        \captionof{figure}{Sistema do \cref{ex:exercício04}}
        \label{fig:exercício04}
    \end{center}
    \begin{enumerate}[label=(\alph*)]
        \item Mostre que a lagrangiana do sistema é
            \begin{equation*}
                L = \frac32 mR^2\dot\theta_1^2 + \frac12 mR^2\dot\theta_2^2 + m R^2\dot\theta_1\dot\theta_2\cos{(\theta_1 - \theta_2)} + 2mgR \cos\theta_1 + mgR\cos\theta_2.
            \end{equation*}
        \item Considerando pequenas oscilações, obtenha os modos normais e respectivas frequências.
        \item Obtenha a solução para a condição inicial \(\theta_1(0) = 0,\) \(\theta_2(0) = \theta_0,\) e \(\dot\theta_1(0) = \dot\theta_2(0) = 0.\)
    \end{enumerate}
\end{exercício}
\begin{proof}[Resolução]
    Tomando o ponto \(O\) como a origem do sistema de coordenadas, a posição da conta é dada por
    \begin{equation*}
        \vetor{r}_2 = \vetor{r}_1 + R\left(\sin\theta_2\vetor{e}_x - \cos\theta_2 \vetor{e}_y\right),
    \end{equation*}
    em que \(\vetor{r}_1 = R\left(\sin\theta_1\vetor{e}_x - \cos\theta_1\vetor{e}_y\right)\) é a posição do centro de massa do anel. As velocidades da conta e do centro de massa do anel são dadas por
    \begin{equation*}
        \dot{\vetor{r}}_1 = R\dot\theta_1\left(\cos\theta_1\vetor{e}_x + \sin\theta_1\vetor{e}_y\right)\quad\text{e}\quad\dot{\vetor{r}}_2 = \dot{\vetor{r}}_1 + R\dot\theta_2\left(\cos\theta_2\vetor{e}_x + \sin\theta_2\vetor{e}_y\right),
    \end{equation*}
    portanto a energia cinética da conta é
    \begin{align*}
        T_2 &= \frac12 m \left[R^2\dot\theta_1^2 + 2\inner*{\dot{\vetor{r}}_1}{R\dot\theta_2\left(\cos\theta_2\vetor{e}_x + \sin\theta_2\vetor{e}_y\right)} + R^2\dot\theta_2^2\right]\\
        &= \frac12 m R^2\left(\dot\theta_1^2 + \dot\theta_2^2\right) + mR^2\dot\theta_1\dot\theta_2 \left(\cos\theta_1\cos\theta_2 + \sin\theta_1 \sin\theta_2\right)\\
        &= \frac12 m R^2\left(\dot\theta_1^2 + \dot\theta_2^2\right) + mR^2\dot\theta_1\dot\theta_2 \cos\left(\theta_1-\theta_2\right).
    \end{align*}
    O momento de inércia do anel pelo eixo que passa por seu centro de massa é \(mR^2\), portanto o momento de inércia pelo ponto \(O\) é \(2mR^2\), pelo teorema dos eixos paralelos. Assim, a energia cinética do anel é dada por \(T_1 = mR^2\dot\theta_1^2\), de forma que a energia cinética do sistema é
    \begin{equation*}
        T = \frac32 m R^2\dot\theta_1^2 + \frac12 m R^2\dot\theta_2^2+ mR^2\dot\theta_1\dot\theta_2 \cos\left(\theta_1-\theta_2\right).
    \end{equation*}

    A energia potencial do sistema é dada por
    \begin{align*}
        V &= mg \inner{\vetor{r}_1}{\vetor{e}_y} + mg\inner{\vetor{r}_2}{\vetor{e}_y}\\
          &= -2 mgR \cos\theta_1 - mgR \cos\theta_2,
    \end{align*}
    portanto a lagrangiana do sistema é
    \begin{equation*}
        L = \frac32 mR^2\dot\theta_1^2 + \frac12 mR^2\dot\theta_2^2 + m R^2\dot\theta_1\dot\theta_2\cos{(\theta_1 - \theta_2)} + 2mgR \cos\theta_1 + mgR\cos\theta_2.
    \end{equation*}

    Para uma configuração de equilíbrio \(q^{(0)} = \left(\theta_1^{(0)}, \theta_2^{(0)}\right)\), devemos ter
    \begin{equation*}
        \diffp{V}{\theta_k}[q^{(0)}] = 0 \implies \theta^{(0)}_k = n_k\pi,
    \end{equation*}
    com \(n_k \in \mathbb{Z}\) e \(k \in \set{1,2}\). Definimos
    \begin{equation*}
        V_{k\ell} = \diffp{V}{\theta_k,\theta_\ell}[q^{(0)}]
    \end{equation*}
    e representamos matricialmente por
    \begin{equation*}
        \left[V_{k\ell}\right] = \begin{bmatrix}
            2mgR(-1)^{n_1} && 0\\
            0 && mgR(-1)^{n_2}
        \end{bmatrix}.
    \end{equation*}
    Para que \(q^{(0)}\) seja um ponto de equilíbrio, devemos ter \(n_1\) e \(n_2\) pares, isto é, \(q^{(0)} \equiv (0,0)\).

    Expandindo a lagrangiana em até segunda ordem de \(\theta_k\) em torno de \(q^{(0)}\), temos
    \begin{equation*}
        L = \frac12T_{k\ell} \dot{\theta_k} \dot{\theta_\ell} - \frac12 V_{k\ell} \theta_k \theta_\ell,
    \end{equation*}
    com
    \begin{equation*}
        [T_{k\ell}] = \begin{bmatrix}
            3mR^2 && mR^2\\
            mR^2 && mR^2
        \end{bmatrix}
        \quad\text{e}\quad[V_{k\ell}] = \begin{bmatrix}
            2mgR && 0\\
            0 && mgR
        \end{bmatrix}.
    \end{equation*}
    Aplicando Euler-Lagrange, temos as equações de movimento dadas por
    \begin{equation*}
        T_{k n} \ddot{\theta}_k + V_{k n} \theta_k = 0.
    \end{equation*}
    Com o ansatz \(\theta_k = \vartheta_k e^{i\omega t}\), temos
    \begin{equation*}
        \left(V_{kn}-\omega^2T_{kn}\right)\vartheta_k e^{i\omega t} = 0.
    \end{equation*}
    Matricialmente,
    \begin{equation*}
        \begin{bmatrix}
            2mgR-3mR^2\omega^2 && - mR^2 \omega^2\\
            -mR^2\omega^2 && mgR - mR^2 \omega^2
        \end{bmatrix}
        \begin{bmatrix}
            \vartheta_1\\\vartheta_2
        \end{bmatrix}
        =
        \begin{bmatrix}
            0\\0
        \end{bmatrix}.
    \end{equation*}
    Para uma solução não trivial, esta matriz deve ter determinante nulo, portanto temos a equação
    \begin{equation*}
        (2mgR - 3mR^2\omega^2)(mgR - mR^2\omega^2) = (mR^2\omega^2)^2 \iff 2R^2\omega^4 - 5gR\omega^2 + 2g^2 = 0,
    \end{equation*}
    cujas soluções são
    \begin{equation*}
        \omega^2 = \frac{5gR \pm 3gR}{4R^2} \implies \omega_+^2 = \frac{2g}{R}\quad\text{e}\quad\omega_-^2 = \frac{g}{2R}.
    \end{equation*}
    Substituindo esses resultados na matriz podemos determinar os autovetores associados à estas frequências de oscilação. Para \(\omega^2 = \omega_+^2\),
    \begin{equation*}
       \begin{bmatrix}
           -4mgR && -2mgR\\
           -2mgR && -mgR
       \end{bmatrix}
       \begin{bmatrix}
           \vartheta^+_1\\\vartheta^+_2
       \end{bmatrix}
       =
       \begin{bmatrix}
           0\\0
       \end{bmatrix}
       \implies
       \begin{bmatrix}
           \vartheta^+_1\\\vartheta^+_2
       \end{bmatrix}
       =
       c_+\begin{bmatrix}
           -1\\2
       \end{bmatrix},
    \end{equation*}
    para uma constante \(c_+ \in \mathbb{C}\) qualquer. Para \(\omega^2 = \omega_-^2\),
    \begin{equation*}
       \begin{bmatrix}
           \frac12mgR && -\frac12mgR\\
           -\frac12mgR && \frac12mgR
       \end{bmatrix}
       \begin{bmatrix}
           \vartheta^-_1\\\vartheta^-_2
       \end{bmatrix}
       =
       \begin{bmatrix}
           0\\0
       \end{bmatrix}
       \implies
       \begin{bmatrix}
           \vartheta^-_1\\\vartheta^-_2
       \end{bmatrix}
       =
       c_-\begin{bmatrix}
           1\\1
       \end{bmatrix},
    \end{equation*}
    para uma constante \(c_- \in \mathbb{C}\) qualquer.

    Desse modo, tomando a parte real das soluções, obtemos os modos normais
    \begin{equation*}
        \begin{bmatrix}
            \theta^+_1(t)\\\theta^+_2(t)
        \end{bmatrix}
        = \alpha \begin{bmatrix}
            -1\\2
        \end{bmatrix}
        \cos\left(\sqrt{\frac{2g}{R}}t + \varphi_+\right)
        \quad\text{e}\quad
        \begin{bmatrix}
            \theta^-_1(t)\\\theta^-_2(t)
        \end{bmatrix}
        = \beta \begin{bmatrix}
            1\\1
        \end{bmatrix}
        \cos\left(\sqrt{\frac{g}{2R}}t + \varphi_-\right),
    \end{equation*}
    com frequências de oscilação \(\sqrt{\frac{2g}{R}}\) e \(\sqrt{\frac{g}{2R}}\). Logo, a solução geral é dada pela combinação linear dos modos normais,
    \begin{equation*}
        \begin{bmatrix}
            \theta_1(t)\\\theta_2(t)
        \end{bmatrix}
        = \alpha \begin{bmatrix}
            -1\\2
        \end{bmatrix}
        \cos\left(\sqrt{\frac{2g}{R}}t + \varphi_+\right)
        + \beta \begin{bmatrix}
            1\\1
        \end{bmatrix}
        \cos\left(\sqrt{\frac{g}{2R}}t + \varphi_-\right),
    \end{equation*}
    para constantes \(\alpha, \beta > 0\) e \(\varphi_+, \varphi_- \in [0, 2\pi]\) determinadas a partir das condições iniciais.

    Para a condição inicial \(\theta_1(0) = 0,\) \(\theta_2(0) = \theta_0,\) e \(\dot\theta_1(0) = \dot\theta_2(0) = 0,\) temos
    \begin{equation*}
        \begin{bmatrix}
            0\\\theta_0
        \end{bmatrix}
        = \alpha \begin{bmatrix}
            -1\\2
        \end{bmatrix}
        \cos\left(\varphi_+\right)
        + \beta \begin{bmatrix}
            1\\1
        \end{bmatrix}
        \cos\left(\varphi_-\right)
        \quad\text{e}\quad
        \begin{bmatrix}
            0\\0
        \end{bmatrix}
        = -\sqrt{\frac{2g}{R}}\alpha \begin{bmatrix}
            -1\\2
        \end{bmatrix}
        \sin\left(\varphi_+\right)
        - \sqrt{\frac{g}{2R}}\beta \begin{bmatrix}
            1\\1
        \end{bmatrix}
        \sin\left(\varphi_-\right).
    \end{equation*}
    Definindo \(A_+ = \alpha \cos\varphi_+, B_+ = \alpha\sin\varphi_+,A_- = \beta \cos\varphi_-,\) e \(B_- = \beta\sin\varphi_-\), temos o sistema de equações lineares e soluções
    \begin{equation*}
        \begin{bmatrix}
            -1 && 0 && 1 && 0\\
            2 && 0 && 1 && 0\\
            0 && \sqrt{\frac{2g}{R}} && 0 && -\sqrt{\frac{g}{2R}}\\
            0 && -2\sqrt{\frac{2g}{R}} && 0 && -\sqrt{\frac{g}{2R}}\\
        \end{bmatrix}
        \begin{bmatrix}
            A_+\\ B_+\\A_-\\B_-
        \end{bmatrix}
        =
        \begin{bmatrix}
            0\\\theta_0\\0\\0
        \end{bmatrix}
        \implies
        \begin{bmatrix}
            A_+\\ B_+\\A_-\\B_-
        \end{bmatrix}
        =
        \begin{bmatrix}
            \frac13 \theta_0\\0\\\frac13\theta_0\\0
        \end{bmatrix},
    \end{equation*}
    de modo que \(\alpha = \beta = \frac13\theta_0\) e \(\varphi_+ = \varphi_- = 0\). Assim,
    \begin{equation*}
        \theta_1(t) = \frac13\theta_0\left[-\cos\left(\sqrt{\frac{2g}{R}}t\right) + \cos\left(\sqrt{\frac{g}{2R}}t\right)\right]\quad\text{e}\quad
        \theta_2(t) = \frac13\theta_0\left[2\cos\left(\sqrt{\frac{2g}{R}}t\right) + \cos\left(\sqrt{\frac{g}{2R}}t\right)\right]
    \end{equation*}
    é a solução para as condições iniciais dadas.
\end{proof}
