\section*{Exercício 4}
\begin{exercício}{Transformação de escala}{exercício04}
    Considere um sistema unidimensional com coordenada generalizada \(q\) cuja lagrangiana é
    \begin{equation*}
        L = \frac12 m \dot{q}^2 - V(q).
    \end{equation*}
    Uma transformação de escala modifica a variável independente tempo bem como a variável dinâmica \(q\) segundo
    \begin{equation*}
        t \to \tilde{t} = \rho t\quad\text{e}\quad q(t) \to \tilde{q}(\tilde{t}) = \rho^d q(\rho t),
    \end{equation*}
    onde a quantidade \(d\) é chamada \emph{dimensão de escala} da variável dinâmica \(q\).
    \begin{enumerate}[label=(\alph*)]
        \item Derive a forma infinitesimal desta transformação.
        \item Qual o valor que a dimensão de escala \(d\) deve ter para que a teoria seja invariante por esta transformação para \(V = 0\)?
        \item Para o valor de \(d\) encontrado, determine a forma mais geral de \(V\) para que a teoria seja invariante por transformações de escala.
        \item Usando o teorema de Noether, obtenha a quantidade conservada \(D\) pela transformação de escala.
    \end{enumerate}
\end{exercício}
\begin{proof}[Resolução]
    Para que a transformação seja a identidade, devemos ter \(\rho = 1\). Deste modo, podemos parametrizar a transformação ao redor da identidade por
    \begin{equation*}
        t \to \tilde{t} = e^\alpha t\quad\text{e}\quad q(t) \to \tilde{q}(\tilde{t}) = e^{\alpha d} q(\rho t),
    \end{equation*}
    para algum \(\alpha \in \mathbb{R}\). Assim, para \(\alpha \ll 1\) temos o gerador infinitesimal desta transformação dado por
    \begin{align*}
        t \to \tilde{t} &= (1 + \alpha)t& q(t) \to \tilde{q}(\tilde{t}) &= (1+d\alpha)q(t + \alpha t),\\
                        &= t + \alpha t& &=(1+d \alpha)q(t) + \alpha t \dot{q}(t).
    \end{align*}

    % Dessa forma, temos \(\diff{\tilde{t}}{t} = 1 + \alpha\) e \(\diff{t}{\tilde{t}} = 1 - \alpha\), portanto
    % \begin{align*}
    %     \diff{\tilde{q}}{\tilde{t}} &= \diff{t}{\tilde{t}} \diff{\tilde{q}}{t} = (1 - \alpha)[(1+d\alpha)\dot{q}(t) + \alpha t\ddot{q}(t) + \alpha \dot{q}(t)]\\
    %                                 &= (1 + d \alpha)\dot{q}(t) + \alpha t \ddot{q}(t).
    % \end{align*}
    Definindo \(X = t,\) \(\psi = d q(t) + t \dot{q}(t)\), temos
    \begin{equation*}
        \tilde{t} = t + \alpha X\quad\text{e}\quad\tilde{q}(\tilde{t}) = q(t) + \alpha \psi.
    \end{equation*}
    Desse modo, para que a ação seja quase-invariante pela transformação, devemos ter
    \begin{equation*}
        L \dot{X} + \psi \diffp{L}{q} + (\dot{\psi} - \dot{X}\dot{q})\diff{L}{\dot{q}} + X \diffp{L}{t} = \dot{G}
    \end{equation*}
    para alguma função \(G = G(q, t)\). Computando diretamente vemos que
    \begin{align*}
        L \dot{X} + \psi \diffp{L}{q} + (\dot{\psi} - \dot{X}\dot{q})\diff{L}{\dot{q}} + X \diffp{L}{t} &= \frac12 m\dot{q}^2 - V(q) + (dq + t\dot{q})(-V'(q)) + (d\dot{q} + t\ddot{q})m\dot{q}\\
                                                                                                                &= \left(d + \frac12\right)m\dot{q}^2 - V(q) - (dq + t\dot{q})V'(q) + m\ddot{q} t\dot{q}\\
                                                                                                                &= \left(d + \frac12\right)m\dot{q}^2 -V(q) - (dq + 2t\dot{q})V'(q),
    \end{align*}
    onde utilizamos a equação de movimento \(m\ddot{q} = -V'(q)\). Notemos que
    \begin{equation*}
        \diff*{\left[tV(q)\right]}{t} = V(q) + t\dot{q}V'(q),
    \end{equation*}
    portanto
    \begin{equation*}
        L \dot{X} + \psi \diffp{L}{q} + (\dot{\psi} - \dot{X}\dot{q})\diff{L}{\dot{q}} + X \diffp{L}{t}  = \left(d + \frac12\right)m\dot{q}^2 + \left[V(q) - dqV'(q)\right] - \diff*{\left[2tV(q)\right]}{t},
    \end{equation*}
    isto é, se \(d = -\frac12\) e o potencial satisfizer a equação diferencial
    \begin{equation*}
        V(q) - dqV'(q) = 0,
    \end{equation*}
    obtemos \(G(q, t) = -2 tV(q)\), logo a ação é quase-invariante por transformações de escala.

    Neste caso, integrando a equação diferencial, obtemos o potencial
    \begin{equation*}
        V(q) = kq^{-2}
    \end{equation*}
    para alguma constante de integração \(k \in \mathbb{R}\). Ainda, pelo \nameref{thm:noether_generalizado}, a quantidade
    \begin{align*}
        D &= hX - \diffp{L}{\dot{q}}\psi + G\\
          &= ht - m\dot{q}\left(-\frac{1}{2}q + t\dot{q}\right) - 2t V(q)\\
          &= ht + \frac12m\dot{q}q - mt \dot{q}^2 - 2t V(q)\\
          &= \frac12 mq\dot{q}-ht
    \end{align*}
    é a integral de movimento associada à transformação de escala.
\end{proof}
