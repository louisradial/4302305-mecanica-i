\documentclass[12pt,a4paper]{article}

% Language and formatting
\usepackage{polyglossia}
\usepackage{csquotes} %fvextra to avoid warning?
\setmainlanguage[variant=brazilian]{portuguese}

% \usepackage[backend=biber, style=alphabetic, sorting=ynt]{biblatex}
% \addbibresource{bibliography.bib}

% \setmainfont{Palatino Linotype}
% \setmathfont{Palatino Linotype}
\usepackage[a4paper, margin=1.5cm]{geometry}
\usepackage{booktabs}

% % title header
% \usepackage{titleps}% http://ctan.org/pkg/titleps
% \makeatletter
% \newpagestyle{main}{% Define page style main
%     \sethead%
%     [\textbf\thepage][][\thechapter.\ \chaptertitle]% [<even-left>][<even-center>][<even-right>]
%     {\thesection.\ \sectiontitle}{}{\textbf\thepage}% {<odd-left>}{<odd-center>}{<odd-right>}
%     \setfoot{}{}{}% {<left>}{<center>}{<right>}
% }
% \pagestyle{main}% Use page style main

% Images
\usepackage{tikz}
\usetikzlibrary{cd}
\usepackage{graphicx, caption, subcaption}
\usepackage{float}

% Math tools
\usepackage{amsfonts, mathtools, amssymb, amsmath, amsthm, enumitem}
\usepackage{newpxtext, newpxmath}
% \numberwithin{equation}{section}
\usepackage[ISO]{diffcoeff}
\usepackage{tensor}
\usepackage{siunitx}

% Misc
\usepackage{luacolor}
\usepackage[breakable]{tcolorbox}

\difdef{fp}{}{
    outer-Ldelim = \left.,
    outer-Rdelim = \right|,
    sub-nudge=0 mu
}
\newcommand\todo[1][!]{{\color{Red} TODO {#1}}}
\difdef{l}{i}{outer-Rdelim = \,, outer-Ldelim=}
\difdef{l}{dn}{style=d^}
\NewDocumentCommand\dli{}{\dl.i.}
\DeclareMathOperator\Riem{Riem}
\DeclareMathOperator\Orb{Orb}
\DeclareMathOperator\Stab{Stab}
\DeclareMathOperator\sgn{sgn}
\DeclareMathOperator\End{End}
\DeclareMathOperator\tr{tr}
\DeclareMathOperator\sech{sech}
\DeclareMathOperator\artanh{artanh}
\DeclareMathOperator\hor{hor}
\DeclareMathOperator\ver{ver}
\DeclarePairedDelimiter\abs{\lvert}{\rvert}
\DeclarePairedDelimiter\norm{\lVert}{\rVert}
\DeclarePairedDelimiterX\inner[2]{\langle}{\rangle}{#1,\mathopen{}#2}
\DeclarePairedDelimiter\set{\{}{\}}

\newenvironment{smallpmatrix}{\left(\begin{smallmatrix}}{\end{smallmatrix}\right)}
\newcommand\ract{\mathbin{\vartriangleleft}}
\newcommand\ractalt{\mathbin{\blacktriangleleft}}
\newcommand\lact{\mathbin{\vartriangleright}}
\newcommand\lactalt{\mathbin{\blacktriangleright}}
\newcommand\ad[1]{\operatorname{ad}_{#1}}
\newcommand\Ad[1]{\operatorname{Ad}_{#1}}
\newcommand\preim[2]{\operatorname{preim}_{#1}{\left(#2\right)}}
\newcommand\id[1]{\operatorname{id}_{#1}}
\newcommand\colorunderline[2]{{\color{#1}\underline{{\color{black}{#2}}}}}
\newcommand\Hom[2][]{\ensuremath{\operatorname{Hom}_{#1}{\left(#2\right)}}}
\newcommand\bundle[3]{\ensuremath{#1 \mathrel{\overset{#2}{\to}} #3}}
\newcommand\smooth[1]{\ensuremath{\mathcal{C}^\infty(#1)}}
\newcommand\sections[1]{\ensuremath{\Gamma\left(#1\right)}}
\newcommand\forms[2][]{\ensuremath{\Lambda^{#1}{\left({#2}\right)}}}
\newcommand\ffamily[3]{\ensuremath{\set*{#1}_{#2}^{#3}}}
\newcommand\family[2]{\ensuremath{\set*{#1}_{#2}}}
\newcommand\vetor[1]{\ensuremath{\boldsymbol{#1}}}
\newcommand\linear{\ensuremath{\mathrel{\tilde{\to}}}}
\newcommand\topology[1]{\ensuremath{\left(#1, \mathcal{O}_{#1}\right)}}
\newcommand\manifold[1]{\ensuremath{\left(#1, \mathcal{O}_{#1}, \mathscr{A}_{#1}\right)}}
\newcommand\restrict[2]{\ensuremath{\left.#1\right\rvert_{#2}}}
\newcommand\bfield[1]{\ensuremath{\diffp*{}{#1}}}
\newcommand\bvec[3][]{\ensuremath{\diffp*{#1}{#2}[#3]}}
\newcommand\bset[3]{\ensuremath{\set*{\diffp*{}{{#1}^1}[#3], \dots, \diffp*{}{{#1}^{#2}}[#3]}}}
\newcommand\pf[2][]{\ensuremath{{#2}_{\ast{#1}}}}
\newcommand\pb[2][]{\ensuremath{{#2}^{\ast}_{#1}}}

% catppuccin (latte)
\definecolor{Rosewater}{RGB}{220,138,120}
\definecolor{Flamingo}{RGB}{221,120,120}
\definecolor{Pink}{RGB}{234,118,203}
\definecolor{Mauve}{RGB}{136,57,239}
\definecolor{Red}{RGB}{210,15,57}
\definecolor{Maroon}{RGB}{230,69,83}
\definecolor{Peach}{RGB}{254,100,11}
\definecolor{Yellow}{RGB}{223,142,29}
\definecolor{Green}{RGB}{64,160,43}
\definecolor{Teal}{RGB}{23,146,153}
\definecolor{Sky}{RGB}{4,165,229}
\definecolor{Sapphire}{RGB}{32,159,181}
\definecolor{Blue}{RGB}{30,102,245}
\definecolor{Lavender}{RGB}{114,135,253}
\definecolor{Text}{RGB}{76,79,105}
\definecolor{Subtext1}{RGB}{92,95,119}
\definecolor{Subtext0}{RGB}{108,111,133}
\definecolor{Overlay2}{RGB}{124,127,147}
\definecolor{Overlay1}{RGB}{140,143,161}
\definecolor{Overlay0}{RGB}{156,160,176}
\definecolor{Surface2}{RGB}{172,176,190}
\definecolor{Surface1}{RGB}{188,192,204}
\definecolor{Surface0}{RGB}{204,208,218}
\definecolor{Base}{RGB}{239,241,245}
\definecolor{Mantle}{RGB}{230,233,239}
\definecolor{Crust}{RGB}{220,224,232}

% References
\usepackage{hyperref}
\usepackage[brazilian,capitalize,nameinlink,noabbrev]{cleveref}
\makeatletter
\hypersetup{
    pdftitle=\@title,
    pdfauthor=\@author,
    colorlinks=true,
    linkcolor=Mauve,
    citecolor=pink,
    filecolor=red,
    urlcolor=blue,
    bookmarksdepth=4
}
\makeatother

% tcolorbox environments
\tcbuselibrary{theorems}
% theorem
\newtcbtheorem[auto counter, crefname={Teorema}{Teoremas}]{theorem}{Teorema}%
{breakable,colback=Mauve!5,colframe=Mauve!95!black,fonttitle=\bfseries}{thm}

% definition
\newtcbtheorem[auto counter, crefname={Definição}{Definições}]{definition}{Definição}%
{breakable, colback=Pink!5,colframe=Pink!95!black,fonttitle=\bfseries}{def}

% proposition
\newtcbtheorem[auto counter, crefname={Proposição}{Proposições}]{proposition}{Proposição}%
{breakable,colback=Lavender!5,colframe=Lavender!95!black,fonttitle=\bfseries}{prop}

% lemma
\newtcbtheorem[auto counter, crefname={Lema}{Lemas}]{lemma}{Lema}%
{breakable,colback=Flamingo!5,colframe=Flamingo!95!black,fonttitle=\bfseries}{lem}

% exercício
\newtcbtheorem[auto counter, crefname={Exercício}{Exercícios}]{exercício}{Exercício}%
{breakable,colback=Sky!5,colframe=Sky!95!black,fonttitle=\bfseries}{ex}

\title{4302305 - Lista de Exercícios V}
\author{Louis Bergamo Radial\\8992822}

\begin{document}
\maketitle

\begin{theorem}{Teorema de Noether}{noether}
    Seja \(L\) uma lagrangiana invariante em relação ao subgrupo a um parâmetro de difeomorfismos \(h^s\), isto é,
    \begin{equation*}
        h^0(q^k(t)) = q^k(t)\quad\text{e}\quad \diffp*{L\left(h^s(q^k), \diffp*{h^s(q^k)}{t}, t\right)}{s} = 0,
    \end{equation*}
    então a quantidade \(\diffp{L}{\dot{q}^k}\diffp{h^s\circ q^k}{s}[s=0]\) é conservada.
\end{theorem}
\begin{proof}
    Sejam \(q^k(t)\) coordenadas generalizadas que satisfazem as equações de Euler-Lagrange. Como \(L\) é invariante por transformações de \(h^s\), segue que \(\tilde{q}^k(s,t) = h^s\circ q^k(t)\) também satisfazem as equações de Euler-Lagrange.

    Ainda pela invariância pelo subgrupo a um parâmetro de difeomorfismos, temos
    \begin{equation*}
        \diffp{L(\tilde{q}^k, \dot{\tilde{q}}^k, t)}{s}[s=0] = \diffp{L}{q^k}\diffp{\tilde{q}^k}{s}[s=0] + \diffp{L}{\dot{q}^k}\diffp{\dot{\tilde{q}}^k}{s}[s=0] = 0.
    \end{equation*}
    Como \(\tilde{q}^k\) satisfazem as equações de Euler-Lagrange, temos
    \begin{equation*}
        \diff*{\left(\diffp{L}{\dot{q}^k}\right)}{t}\diffp{\tilde{q}^k}{s}[s=0] + \diffp{L}{\dot{q}^k}\diffp*{\diffp{\tilde{q}}{t}}{s}[s=0] = \diff*{\left(\diffp{L}{\dot{q}^k}\diffp{\tilde{q}^k}{s}[s=0]\right)}{t} = 0,
    \end{equation*}
    como queríamos demonstrar.
\end{proof}

\begin{theorem}{Teorema de Noether generalizado}{noether_generalizado}
    Se a ação é \emph{quase-invariante} sob a transformação infinitesimal
    \begin{align*}
        t &\to \tilde{t} = t + \epsilon X(q(t), t)&q^k(t) &\to \tilde{q}^k(\tilde{t}) = q^k(t) + \epsilon \psi^k(q(t), \dot{q}(t), t),
    \end{align*}
    isto é, se existe uma função \(G = G(q(t), t)\) tal que
    \begin{equation*}
        \Delta S = \int_{\tilde{t}_A}^{\tilde{t}_B} \dli{\tilde{t}} \tilde{L} - \int_{t_A}^{t_B} \dli{t} \left(L + \epsilon \diff*{G(q(t), t)}{t}\right) = 0,
    \end{equation*}
    onde \(\tilde{L} = L\left(\tilde{q}(\tilde{t}), \diff{\tilde{q}}{\tilde{t}}, \tilde{t}\right)\) e \(L = L\left(q(t), \diff{q}{t}, t\right)\), então a quantidade
    \begin{equation*}
        I = hX - \diffp{L}{\dot{q}^k}\psi^k + G
    \end{equation*}
    é a integral de movimento associada à simetria, com a função energia dada por \(h = \diffp{L}{\dot{q}^k}\dot{q}^k - L\).
\end{theorem}
\begin{proof}
    Notemos que
    \begin{equation*}
        \diff{\tilde{t}}{t} = 1 + \epsilon \dot{X}\quad{\text{e}}\quad\diff{t}{\tilde{t}} = 1 - \epsilon\dot{X}
    \end{equation*}
    em primeira ordem em \(\epsilon\). Assim, as velocidades generalizadas sob a transformação infinitesimal são dadas por
    \begin{equation*}
        \diff{\tilde{q}^k}{\tilde{t}} = \diff{t}{\tilde{t}}\diff{\tilde{q}}{t} = (1 - \epsilon \dot{X})(\dot{q}^k + \epsilon \dot{\psi^k}) = \dot{q}^k + \epsilon \xi^k,
    \end{equation*}
    onde definimos \(\xi^k = \dot{\psi^k} - \dot{X}\dot{q}^k\). Com isso a lagrangiana avaliada após a transformação é
    \begin{align*}
        \tilde{L} &= L(q(t) + \epsilon \psi, \dot{q}(t) + \epsilon\xi, t + \epsilon X)\\
                  &= L + \epsilon \left(\psi^k \diffp{L}{q^k} + \xi^k \diffp{L}{\dot{q}^k} + X\diffp{L}{t}\right).
    \end{align*}

    Se a ação é quase-invariante por esta transformação, temos
    \begin{align*}
        \Delta S &= \int_{\tilde{t}_A}^{\tilde{t}_B}\dli{\tilde{t}} \tilde{L} - \int_{t_A}^{t_B} \dli{t} \left(L + \epsilon \dot{G}\right)\\
                 &= \int_{t_A}^{t_B} \dli{t} \left(\tilde{L}\diff{\tilde{t}}{t} - L - \epsilon \dot{G}\right)\\
                 &= \epsilon\int_{t_A}^{t_B} \dli{t} \left[L\dot{X} + \psi^k \diffp{L}{q^k} + \xi^k\diffp{L}{\dot{q}^k} + X \diffp{L}{t} - \dot{G}\right] = 0.
    \end{align*}
    Portanto, a condição necessária e suficiente para que a ação seja quase-invariante por esta transformação é
    \begin{equation*}
        \dot{G} = L \dot{X} + \psi^k \diffp{L}{q^k} + (\dot{\psi}^k - \dot{X}\dot{q}^k)\diffp{L}{\dot{q}^k} + X\diffp{L}{t},
    \end{equation*}
    isto é, o lado direito desta equação deve ser igual à alguma derivada total em relação ao tempo de uma função das posições e do tempo.

    Pela definição da função energia temos
    \begin{align*}
        \diff{G}{t} &= \left(L - \diffp{L}{\dot{q}^k}\dot{q}^k\right)\dot{X} + X\diffp{L}{t} + \psi^k\diff*{\left(\diffp{L}{\dot{q}^k}\right)}{t} + \dot{\psi}^k\diffp{L}{\dot{q}^k}\\
                    &= -h \dot{X} - \diff{h}{t}X + \diff*{\left(\psi^k\diffp{L}{\dot{q}^k}\right)}{t}\\
                    &= \diff*{\left(-hX + \psi^k\diffp{L}{\dot{q}^k}\right)}{t},
    \end{align*}
    onde utilizamos as equações de Euler-Lagrange. Desse modo, temos que
    \begin{equation*}
        \diff{I}{t} = 0,
    \end{equation*}
    onde \(I = hX - \psi^k\diffp{L}{\dot{q}^k} + G\), como desejado.
\end{proof}

\section*{Exercício 1}
\begin{lemma}{Partícula em um campo eletromagnético externo}{lagrangiana_lorentz}
    A lagrangiana de uma partícula de massa \(m\) e carga \(e\) em um campo eletromagnético externo definido pelo potencial escalar \(\phi\) e pelo potencial vetor \(\vetor{A}\) é dada por
    \begin{equation*}
        L = \frac12 m g_{ij} \dot{x}^i \dot{x}^j - e \phi + e g_{ij} A^i \dot{x}^j,
    \end{equation*}
    onde \(g_{ij}\) é o tensor métrico Euclidiano.
\end{lemma}
\begin{proof}
    Uma partícula de massa \(m\) e carga \(e\) em um campo eletromagnético externo está sujeita à força de Lorentz dada por
    \begin{equation*}
        \vetor{F} = e\left(\vetor{E} + \vetor{v}\times\vetor{B}\right),
    \end{equation*}
    onde \(\vetor{E}\) é o campo elétrico e \(\vetor{B}\) é o campo magnético.

    Podemos definir os campos elétrico e magnético a partir de um potencial escalar \(\phi\) e um potencial vetor \(\vetor{A}\),
    \begin{equation*}
        \vetor{E} = -\vetor{\nabla}\phi - \diffp{\vetor{A}}{t}\quad\text{e}\quad\vetor{B} = \vetor{\nabla} \times \vetor{A},
    \end{equation*}
    de modo que as equações de Maxwell ainda sejam satisfeitas. Neste caso, a força de Lorentz é dada por
    \begin{equation*}
        \vetor{F} = e\left(-\vetor{\nabla}\phi - \diffp{\vetor{A}}{t} + \vetor{v}\times (\vetor{\nabla}\times \vetor{A})\right).
    \end{equation*}

    Notemos que
    \begin{equation*}
        \diff{\vetor{A}}{t} = \diffp{\vetor{A}}{x^i}\dot{x}^i + \diffp{\vetor{A}}{t} \implies -\diffp{A}{t} = \inner{\vetor{v}}{\vetor{\nabla}}\vetor{A} - \diff{\vetor{A}}{t}
    \end{equation*}
    e que
    \begin{equation*}
        \vetor{v} \times (\vetor{\nabla}\times \vetor{A}) = \vetor{\nabla}\inner{\vetor{v}}{\vetor{A}} - \inner{\vetor{v}}{\vetor{\nabla}}\vetor{A},
    \end{equation*}
    uma vez que \(\diffp{\vetor{v}}{x^i} = 0\). Dessa forma, segue que
    \begin{align*}
        \vetor{F} &= e\left(-\vetor{\nabla}\phi - \diff{\vetor{A}}{t} + \vetor{\nabla}\inner{\vetor{v}}{\vetor{A}}\right)\\
                  &= e \left[- \vetor{\nabla}(\phi - \inner{\vetor{v}}{\vetor{A}}) - \diff{\vetor{A}}{t}\right].
    \end{align*}

    Definindo o operador \(\vetor{\nabla}_{\vetor{v}} = \vetor{e}_x\diffp*{}{v_x} + \vetor{e}_y\diffp*{}{v_y} + \vetor{e}_z\diffp*{}{v_z}\) e notando que tanto \(\vetor{A}\) quanto \(\phi\) não dependem das velocidades, temos
    \begin{equation*}
        -\vetor{A} = \vetor{\nabla}_{\vetor{v}}\left(\phi - \inner{\vetor{v}}{\vetor{A}}\right),
    \end{equation*}
    de modo que a força de Lorentz seja dada por
    \begin{equation*}
        \vetor{F} = e \left[- \vetor{\nabla}(\phi - \inner{\vetor{v}}{\vetor{A}}) + \diff*{\vetor{\nabla}_{\vetor{v}}\left(\phi - \inner{\vetor{v}}{\vetor{A}}\right)}{t}\right].
    \end{equation*}

    Isto é, mostramos que a força de Lorentz resulta de um potencial generalizado \(U = e\phi - e\inner{\vetor{v}}{\vetor{A}}\)
    \begin{equation*}
        F_k = - \diffp{U}{x^k} + \diff*{\left(\diffp{U}{\dot{x}^k}\right)}{t},
    \end{equation*}
    portanto a lagrangiana é dada por
    \begin{align*}
        L &= \frac12 m\inner{\vetor{v}}{\vetor{v}} - e\phi + e\inner{\vetor{v}}{\vetor{A}}\\
          &= \frac12 mg_{ij} \dot{x}^i \dot{x}^j - e\phi + eg_{ij}A^i\dot{x}^j,
    \end{align*}
    o que conclui a demonstração.
\end{proof}

\begin{exercício}{Movimento em um campo magnético uniforme}{exercício01}
    Uma partícula de massa \(m\) move-se na presença de um campo magnético constante \(\vetor{B} = B \vetor{e}_z\).
    \begin{enumerate}[label=(\alph*)]
        \item Mostre que o potencial vetor \(\vetor{A} = \frac{B}{2}(-y\vetor{e}_x + x \vetor{e}_y)\) está associado a este campo magnético.
        \item Utilizando o formalismo Lagrangiano obtenha a equação de movimento desta partícula.
        \item Obtenha a trajetória desta partícula utilizando a condição inicial que \(\vetor{r}(0) = \vetor{0}\) e que \(\vetor{v}(0) = a\vetor{e}_x + b\vetor{e}_y,\) onde \(a\) e \(b\) são constantes.
    \end{enumerate}
\end{exercício}
\begin{proof}[Resolução]
    Notemos que
    \begin{equation*}
        \vetor{\nabla} \times \left(- y\vetor{e}_x + x\vetor{e}_y\right) = 2 \vetor{e}_z,
    \end{equation*}
    portanto o potencial vetor \(\vetor{A}\) dado é associado ao campo magnético uniforme \(\vetor{B}\).

    Pelo \cref{lem:lagrangiana_lorentz}, a lagrangiana de uma partícula de carga \(Q\) e massa \(m\) em um campo eletromagnético é dada por
    \begin{equation*}
    L = \frac12 m\inner{\vetor{v}}{\vetor{v}} - Q\phi + Q \inner{\vetor{v}}{\vetor{A}},
    \end{equation*}
    onde \(\vetor{v}\) é a velocidade da partícula e \(\phi\) é o potencial escalar. Neste caso, como não há a presença de um campo elétrico, e como \(\diffp{\vetor{A}}{t} = 0\), segue que \(\phi\) é constante. Desse modo, a lagrangiana pode ser escrita como
    \begin{equation*}
        L(q^1,\dot{q}^1, q^2, \dot{q}^2) = \frac12 m g_{ij}\dot{q}^i \dot{q}^j + \frac{QB}{2} \epsilon_{ij}q^i\dot{q}^j,
    \end{equation*}
    onde \(q^1 = x\), \(q^2 = y\), \(\epsilon_{ij}\) é o símbolo de Levi-Civita, e \(g_{ij}\) é o tensor métrico Euclidiano. Assim, pelas equações de Euler-Lagrange temos
    \begin{align*}
        \diff*{\left(\diffp{L}{\dot{q}^k}\right)}{t} - \diffp{L}{q^k} = 0 &\implies \diff*{\left(\frac12 mg_{ij} \delta^i_k \dot{q}^j + \frac12 g_{ij} \dot{q}^i\delta^j_k + \frac{QB}{2}\epsilon_{ij} q^i \delta^j_k\right)}{t} - \frac{QB}{2} \epsilon_{ij}\delta^i_k \dot{q}^j = 0\\
                                                                          &\implies \diff*{\left(mg_{kj} \dot{q}^j + \frac{QB}{2}\epsilon_{ik} q^i \right)}{t} - \frac{QB}{2} \epsilon_{kj}\dot{q}^j = 0\\
                                                                          &\implies mg_{kj} \ddot{q}^j + \frac{QB}{2}\epsilon_{ik} \dot{q}^i - \frac{QB}{2} \epsilon_{kj}\dot{q}^j = 0\\
                                                                          &\implies m\ddot{q}_k + QB\epsilon_{ik} \dot{q}^i = 0.
    \end{align*}
    De forma explícita, temos as equações de movimento
    \begin{equation*}
        \begin{cases}
            \ddot{q}_1 = \omega\dot{q}_2\\
            \ddot{q}_2 = -\omega\dot{q}_1,
        \end{cases} \implies
        \begin{cases}
            \ddot{x} = \omega\dot{y}\\
            \ddot{y} = -\omega\dot{x},
        \end{cases}
    \end{equation*}
    onde \(\omega = \frac{QB}{m}\).

    Integrando a primeira equação em relação ao tempo no intervalo \([0,t]\), temos
    \begin{equation*}
        \dot{x}(t) - \dot{x}(0) = \omega(y(t) - y(0)) \implies \dot{x}(t) = a + \omega y(t).
    \end{equation*}
    Substituindo na segunda equação, temos a equação diferencial linear não homogênea
    \begin{equation*}
        \ddot{y} + \omega^2 y = - \omega a,
    \end{equation*}
    cuja solução é
    \begin{equation*}
        y(t) = \alpha \cos(\omega t) + \beta \sin(\omega t) - \frac{a}{\omega},
    \end{equation*}
    para constantes de integração \(\alpha, \beta\). Como \(y(0) = 0\) e \(\dot{y}(0) = b\), temos
    \begin{equation*}
        y(t) = \frac{b}{\omega}\sin(\omega t) -  \frac{a}{\omega}\left[1 - \cos(\omega t)\right].
    \end{equation*}
    Assim,
    \begin{equation*}
        \dot{x}(t) = b\sin(\omega t) +  a \cos(\omega t) \implies x(t) = \frac{b}{\omega}\left[1 - \cos(\omega t)\right]+\frac{a}{\omega} \sin(\omega t).
    \end{equation*}

    Deste modo, a partícula tem posição dada por
    \begin{align*}
        \vetor{r}(t) &= \vetor{r}_0 + \left[\frac{a}{\omega} \sin(\omega t) - \frac{b}{\omega}\cos(\omega t)\right]\vetor{e}_x + \left[\frac{b}{\omega} \sin(\omega t) + \frac{a}{\omega}\cos(\omega t)\right]\vetor{e}_y\\
                     &= \vetor{r}_0 + \rho\left[\cos(\varphi - \omega t)\vetor{e}_x + \sin(\varphi-\omega t)\vetor{e}_y\right],
    \end{align*}
    onde \(\vetor{r}_0 = \frac{b}{\omega}\vetor{e}_x - \frac{a}{\omega}\vetor{e}_y\), \(\rho = \sqrt{\frac{a^2 + b^2}{\omega^2}}\) e \(\varphi \in [0, 2\pi]\) é tal que
    \begin{equation*}
        \rho\cos\varphi = -\frac{b}{\omega}\quad\text{e}\quad\rho\sin\varphi = \frac{a}{\omega}.
    \end{equation*}
    Notemos que
    \begin{equation*}
        \inner{\vetor{r}(t) - \vetor{r}_0}{\vetor{r}(t) - \vetor{r}_0} = \rho^2,
    \end{equation*}
    portanto a a trajetória da partícula descreve um círculo de raio \(\rho\) centrado em \(\vetor{r}_0\) com frequência angular constante \(\omega\).
\end{proof}

\section*{Exercício 2}

\section*{Exercício 3}
\begin{exercício}{Geodésicas do cone}{exercício3}
    Obtenha as geodésicas de um cone dado por \(\theta = \alpha\) em coordenadas esféricas.
\end{exercício}
\begin{proof}[Resolução]

\end{proof}

\begin{exercício}{Massa em uma cunha}{exercício4}
    Uma cunha de massa \(M\) repousa sobre um plano horizontal. O ângulo do plano inclinado com a horizontal é \(\alpha\). Um corpo de massa \(m\) é colocado sobre o plano inclinado com seu centro de massa a uma altura \(h\) deste. Desprezando o atrito e usando o formalismo lagrangiano:
    \begin{enumerate}[label=(\alph*)]
        \item obtenha a lagrangiana que descreve o sistema;
        \item obtenha as equações de movimento;
        \item obtenha a solução para o movimento da cunha e do corpo de massa \(m\) assumindo que no instante inicial o corpo e a cunha encontram-se parados;
        \item há alguma quantidade conservada?
    \end{enumerate}
\end{exercício}
\begin{proof}[Resolução]

\end{proof}

\section*{Exercício 5}
\begin{exercício}{Massas oscilando em um eixo livre}{exercício5}
    Dois corpos idênticos de massa \(m\) podem mover-se sem atrito ao longo de uma haste e de forma simétrica como mostra a \cref{fig:teste}.
    \begin{center}
        \includegraphics[width=0.3\linewidth]{exercício05.png}
        \captionof{figure}{Sistema do \cref{ex:exercício5}}
        \label{fig:teste}
    \end{center}
    A massa da barra é desprezível e pode rodar livremente em torno do ponto \(O\). Cada massa \(m\) está conectada à origem por uma mola de constante elástica \(\frac12m\omega_0^2\).
    \begin{enumerate}[label=(\alph*)]
        \item Obtenha a lagrangiana que descreve o sistema.
        \item Obtenha as equações de movimento.
        \item Há alguma quantidade conservada? Interprete.
    \end{enumerate}
\end{exercício}
\begin{proof}[Resolução]
    Como os corpos se encontram ao longo da haste, temos
    \begin{equation*}
        \vetor{r}_1 = r_1\vetor{e}_r\quad\text{e}\quad \vetor{r}_2 = - r_2 \vetor{e}_r,
    \end{equation*}
    onde \(\vetor{e}_r = \cos\phi\sin\theta \vetor{e}_x + \sin\phi\sin\theta \vetor{e}_y + \cos\theta \vetor{e}_z\). Segue que a energia potencial do sistema é dada por
    \begin{equation*}
        U = \frac14 m \omega_0^2\left(r_1^2 + r_2^2\right) + mg\left(r_1 - r_2\right)\cos\theta
    \end{equation*}
    e a energia cinética por
    \begin{equation*}
        T = \frac12 m\left[\dot{r}_1^2 + \dot{r}_2^2 + \left(r_1^2 + r_2^2\right)\left(\dot{\theta}^2 + \dot{\phi}\sin^2\theta\right)\right].
    \end{equation*}
    Utilizando o vínculo de que o movimento ocorre de forma simétrica, \(r \equiv r_1 = r_2\), temos a lagrangiana
    \begin{equation*}
        L = m\left[\dot{r}^2 + r^2\left(\dot{\theta}^2 + \dot{\phi}^2\sin^2\theta\right)\right] - \frac12 m \omega_0^2r^2
    \end{equation*}
    para descrever o sistema.

    Pelas equações de Euler-Lagrange, temos
    \begin{equation*}
        \left\{\begin{aligned}
                \diff*{\left(\diffp{L}{\dot{r}}\right)}{t} - \diffp{L}{r} = 0 &\implies 2m\ddot{r} - 2mr\left(\dot{\theta}^2 + \dot\phi^2\sin^2\theta\right) + m\omega_0^2 r = 0\\
                \diff*{\left(\diffp{L}{\dot{\theta}}\right)}{t} - \diffp{L}{\theta} = 0 &\implies 2mr^2\ddot{\theta} + 4mr\dot{r}\dot{\theta} - 2mr^2\dot\phi^2\sin\theta\cos\theta = 0\\
                \diff*{\left(\diffp{L}{\dot{\phi}}\right)}{t} - \diffp{L}{\phi} = 0 &\implies \diff*{\left(2mr^2\dot\phi\sin^2\theta\right)}{t} = 0
        \end{aligned}\right.
    \end{equation*}
    Como a lagrangiana não depende de forma explícita do tempo e como \(\phi\) é uma coordenada cíclica temos que a energia e a componente \(\vetor{e}_z\) do momento angular do sistema,
    \begin{equation*}
        E = T + U\quad\text{e}\quad J_z = 2mr^2\dot\phi\sin^2\theta
    \end{equation*}
    são conservadas. De fato, o momento angular do sistema é dado por
    \begin{align*}
        \vetor{J} &= [\vetor{r}_1, m\dot{\vetor{r}}_1] + [\vetor{r}_2, m\dot{\vetor{r}}_2]\\
                  &= 2mr\left[\vetor{e}_r, \dot{r}\vetor{e}_r + r\dot{\theta}\vetor{e}_\theta + r\sin\theta \vetor{e}_\phi\right]\\
                  &= 2mr^2 \left(\dot\theta \vetor{e}_\phi - \dot\phi\sin\theta \vetor{e}_\theta\right),
    \end{align*}
    portanto
    \begin{align*}
        J_z &= \inner{\vetor{J}}{\vetor{e}_z}\\
            &= 2mr^2\inner{\dot\theta \vetor{e}_\phi - \dot\phi\sin\theta \vetor{e}_\theta}{\vetor{e}_z}\\
            &= 2mr^2\dot\phi\sin^2\theta,
    \end{align*}
    que é a quantidade conservada afirmada.
\end{proof}

\end{document}
