\section*{Exercício 3}
\textbf{Considere um corpo submetido a um potencial central
    \begin{equation*}
        U(r) = -\frac{\alpha}{r} + \frac{\beta}{r^2}
    \end{equation*}
onde \(\alpha,\beta>0.\)}

\textbf{(a) Existem órbitas circulares? Qual a condição para que isso ocorra?}

Consideremos o potencial efetivo
\begin{align}
    U_\mathrm{ef}(r) &= -\frac{\alpha}{r} + \frac{\beta}{r^2} + \frac{L^2}{2\mu r^2}\\
                     &= -\alpha r^{-1} + \frac{2\mu\beta + L^2}{2\mu}r^{-2}
\end{align}
Temos
\begin{equation}
    \diff{U_\mathrm{ef}}{r} = \alpha r^{-2} - \frac{2\mu\beta + L^2}{\mu}r^{-3}\quad\text{e}\quad\diff[2]{U_\mathrm{ef}}{r} = -\alpha r^{-3} + 3\frac{2\beta\mu + L^2}{\mu}r^{-4}.
\end{equation}
Notemos que \(\diff{U_\mathrm{ef}}{r} = 0\) apenas para \(r = r_0 \equiv \frac{2\beta\mu +L^2}{\mu \alpha}\). Ainda, temos
\begin{equation}
    U_\mathrm{ef}(r_0) \equiv U_0 = -\frac{\alpha}{2r_0}\quad\text{e}\quad\diff[2]{U_\mathrm{ef}}{r}[r=r_0] = \alpha r_0^3 > 0,
\end{equation}
isto é, o valor mínimo do potencial efetivo é \(U_0\), que ocorre em \(r = r_0\).

Substituindo \(\alpha r_0 = \frac{2\beta \mu + L^2}{\mu}\) na expressão do potencial efetivo, podemos escrever
\begin{align}
    U_\mathrm{ef}(r) &= -\alpha r^{-1} + \frac{\alpha r_0}{2}r^{-2}\\
                     &= -\frac{\alpha}{r_0} \left(\frac{r}{r_0}\right)^{-1} + \frac{\alpha r_0}{2r_0^2}\left(\frac{r}{r_0}\right)^{-2}\\
                     &= U_0\left[2\left(\frac{r}{r_0}\right)^{-1} - \left(\frac{r}{r_0}\right)^{-2}\right].
\end{align}

\begin{figure}[H]
    \centering
    \begin{tikzpicture}[domain=0.41:12,smooth,samples=100]
        \draw[->] (0.0,0) -- (12.2,0) node[right] {$r$};
        \draw[->] (0,-2.2) -- (0,2.2) node[left] {$U_{\mathrm{ef}}(r)$};
        \draw[color=Mauve, very thick]    plot (\x,{-4/\x + 2/(\x^2)});
        \draw[dashed,gray, thin] (1,0) node[below right]{\(r_0\)} -- (1, -2);
        \draw[dashed,gray, thin] (0,-2) node[left]{\(U_0\)} -- (1, -2);
        \draw[dashed,gray, thin] (0,{-240/169}) node[left]{\(E\)} -- (12,{-240/169});
        \draw[dashed,gray, thin] (0.65,0) node[above]{\(r_1\)} -- (0.65, {-240/169});
        \draw[dashed,gray, thin] ({13/6},0) node[above]{\(r_2\)} -- ({13/6}, {-240/169});
    \end{tikzpicture}
    \caption{Potencial efetivo para o potencial central \(U(r) = -\frac{\alpha}{r} + \frac{\beta}{r^2}\).}
\end{figure}

Se o sistema tiver energia total \(E \geq U_0,\) um ponto de retorno satisfaz
\begin{equation}
    U_\mathrm{ef}(r) = E \implies \frac{\lambda r^2 - 2r_0 r + r_0^2}{r^2} = 0,
\end{equation}
com \(\lambda = \frac{E}{U_0}.\) Caso \(\lambda = 0,\) há apenas um ponto de retorno em \(r = \frac{r_0}{2}\), de modo que a trajetória não é limitada. Caso \(\lambda \neq 0\), temos
\begin{equation}
    r = \frac{r_0}{1 \pm \epsilon},
\end{equation}
onde \(\epsilon = \sqrt{1 - \lambda} \neq 1\). No caso em que \(E < 0,\) temos \(1 > \epsilon \geq 0\), portanto há dois pontos de retorno
\begin{equation}
    r_1 = \frac{r_0}{1+\epsilon}\quad\text{e}\quad r_2 = \frac{r_0}{1 - \epsilon},
\end{equation}
de modo que a trajetória é limitada, com o caso especial de \(E = U_0\), em que a trajetória é circular \(r_1 = r_2 = r_0\). No caso em que \(E > 0\), temos \(\epsilon > 1\), de modo que há apenas um ponto de retorno
\begin{equation}
    r = \frac{r_0}{1+\epsilon},
\end{equation}
uma vez que \(r > 0\), portanto a trajetória não é limitada.

\textbf{(b) Dada a energia do corpo \(E < 0\), obtenha a sua trajetória.}

Pelo \cref{lem:trajetória}, temos
\begin{align}
    \pm \dl\theta &= \frac{L}{\sqrt{2\mu}}\cdot\frac{\dl{r}}{r^2\sqrt{E - 2U_0\left(\frac{r}{r_0}\right)^{-1} + U_0\left(\frac{r}{r_0}\right)^{-2}}}\\
                  &= -\frac{L}{\sqrt{2\mu r_0^2}}\cdot\frac{\dl{\rho}}{\sqrt{E - 2U_0\rho + U_0 \rho^2}}\\
                  &= -\frac{L}{\sqrt{\mu \alpha r_0}} \cdot \frac{\dl{\rho}}{\sqrt{2\rho - \rho^2 - \lambda}}
\end{align}
onde \(\rho = \frac{r_0}{r}\). Recordando que \(\mu \alpha r_0 = 2\mu\beta + L^2\) e notando que
\begin{align}
    2\rho - \rho^2 - \lambda &= \epsilon^2 - (\rho - 1)^2\\
                             &= \epsilon^2\left[1 - \left(\frac{\rho - 1}{\epsilon}\right)^2\right],
\end{align}
obtemos
\begin{align}
    \pm\sqrt{1 + \frac{2\mu\beta}{L^2}}\dl\theta &= -\frac{\dl \rho}{\epsilon} \left[1 - \left(\frac{\rho - 1}{\epsilon}\right)\right]^{-\frac12}\\
                                                 &= -\frac{\dl{\xi}}{\sqrt{1 - \xi^2}},
\end{align}
com \(\xi = \frac{\rho - 1}{\epsilon}\).

Integrando e fazendo as devidas substituições, segue que
\begin{equation}
    \pm \sqrt{1 + \frac{2\mu \beta}{L^2}}\left(\theta - \theta_i\right) = \arccos{\left(\frac{\frac{r_0}{r(\theta)}-1}{\epsilon}\right)} - \arccos{\left(\frac{\frac{r_0}{r_i}-1}{\epsilon}\right)},
    \label{eq:integral3}
\end{equation}
onde \(r(\theta) = r_i\). Seja
\begin{equation}
    \phi = \sqrt{1+\frac{2\mu\beta}{L^2}}\theta_i \mp \arccos{\left(\frac{\frac{r_0}{r_i}-1}{\epsilon}\right)},
\end{equation}
respeitando a escolha de sinal na \cref{eq:integral3}, de modo que
\begin{equation}
    \frac{\frac{r_0}{r(\theta)}-1}{\epsilon} = \cos\left(\sqrt{1 + \frac{2\mu\beta}{L^2}}\theta - \phi\right).
\end{equation}
Isolando \(r(\theta)\), obtemos a trajetória
\begin{align}
    r(\theta) &= \frac{r_0}{1 + \epsilon \cos{\left(\sqrt{1 + \frac{2\mu\beta}{L^2}}\theta - \phi\right)}}\\
              &= \frac{\frac{2 \beta \mu + L^2}{\mu \alpha}}{1 + \sqrt{1 + \frac{2E(2\beta\mu + L^2)}{\mu\alpha^2}}\cos{\left(\sqrt{1+\frac{2\mu \beta}{L^2}}\theta - \phi\right)}}
\end{align}
para este potencial.
