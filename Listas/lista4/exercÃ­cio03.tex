\section*{Exercício 3}
\begin{exercício}{Geodésicas do cone}{exercício3}
    Obtenha as geodésicas de um cone dado por \(\theta = \alpha\) em coordenadas esféricas.
\end{exercício}
\begin{proof}[Resolução]
    Em \(\mathbb{R}^3\), a métrica em uma superfície \(S\) é induzida pela métrica Euclidiana, isto é, em um ponto \(p\in S\) a métrica é obtida pela restrição do produto interno de \(\mathbb{R}^3\) ao espaço tangente \(T_pS\). Dessa forma, a métrica no cone é dada por
    \begin{equation*}
        \dl[2]{s} = \dl[2]{r} + r^2\sin^2\alpha\dl[2]\phi,
    \end{equation*}
    utilizando coordenadas esféricas, isto é, as componentes da métrica são \(g_{11} = 1\), \(g_{22} = r^2\sin^2\alpha\), e todas as outras são identicamente nulas.

    Assim, os coeficientes da conexão são dados por
    \begin{equation*}
        \Gamma\indices{^k_{ij}} = \frac12 g^{k\ell}\left(\partial_i g_{\ell j} + \partial_j g_{\ell i} - \partial_\ell g_{ij}\right),
    \end{equation*}
    obtendo
    \begin{equation*}
        \Gamma\indices{^1_{22}} = - r\sin^2\alpha\quad\text{e}\quad \Gamma\indices{^2_{12}} = \frac1r = \Gamma\indices{^2_{21}},
    \end{equation*}
    com os outros termos todos nulos. Portanto, as geodésicas de um cone são soluções do sistema de equações diferenciais
    \begin{equation*}
        \left\{\begin{aligned}
                &\ddot{r} - r\dot{\phi}^2\sin^2\alpha = 0\\
                &\ddot{\phi} + \frac2r \dot{r}\dot{\phi} =0
        \end{aligned}\right..
    \end{equation*}
\end{proof}
