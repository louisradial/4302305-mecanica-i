\section*{Exercício 2}
\begin{exercício}{Energia de uma partícula em um campo eletromagnético externo}{exercício02}
    Uma partícula encontra-se na presença de um campo eletromagnético independente do tempo. Utilizando o formalismo lagrangiano, obtenha a energia do sistema.
\end{exercício}
\begin{proof}[Resolução]
    Para uma lagrangiana \(L\) qualquer, a função energia \(h\) é definida por
    \begin{equation*}
        h = \diffp{L}{\dot{q}^k}\dot{q}^k - L,
    \end{equation*}
    com a propriedade
    \begin{equation*}
        \diff{h}{t} = -\diffp{L}{t}.
    \end{equation*}

    Como a lagrangiana para uma partícula de massa \(m\) e carga \(Q\) em um campo eletromagnético externo com potencial escalar \(\phi\) e potencial vetor \vetor{A} é dada por
    \begin{equation*}
        L = \frac12 m g_{ij} \dot{x}^i \dot{x}^j - Q\phi + \frac{Q}{c} g_{ij} A^i \dot{x}^j,
    \end{equation*}
    onde \(x^i\) são as suas coordenadas cartesianas e \(g_{ij}\) o tensor métrico Euclidiano, a função energia neste caso é dada por
    \begin{align*}
        h &= \left(mg_{ij} \dot{x}^i \delta^j_k + \frac{Q}{c} g_{ij} A^i \delta^j_k\right)\dot{x}^k - \left( \frac12 m g_{ij} \dot{x}^i \dot{x}^j - Q\phi +\frac{Q}{c} g_{ij} A^i \dot{x}^j\right)\\
          &= m g_{ik}\dot{x}^i \dot{x}^k + \frac{Q}{c}g_{ik}A^i\dot{x}^k - \left(\frac12 m g_{ik} \dot{x}^i \dot{x}^k - Q\phi + \frac{Q}{c}g_{ik} A^i \dot{x}^k\right)\\
          &= \frac12 m g_{ik} \dot{x}^i\dot{x}^k + Q\phi\\
          &= \frac12 m \inner{\vetor{v}}{\vetor{v}} + Q\phi.
    \end{align*}
    No caso em que o campo eletromagnético não depende do tempo, temos que a função energia é constante no tempo.
\end{proof}
