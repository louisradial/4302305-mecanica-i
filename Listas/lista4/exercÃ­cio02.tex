\section*{Exercício 2}
\begin{lemma}{Função energia}{energia}
    Para uma lagrangiana \(L = L(q^1, \dots, q^n, \dot{q}^1, \dots, \dot{q}^n, t)\), a função energia \(h\) é definida por
    \begin{equation*}
        h = \diffp{L}{\dot{q}^k}\dot{q}^k - L,
    \end{equation*}
    com a propriedade
    \begin{equation*}
        \diff{h}{t} = -\diffp{L}{t}.
    \end{equation*}
\end{lemma}
\begin{proof}
    Computemos a derivada total em relação ao tempo para a função energia. Temos
    \begin{align*}
        \diff{h}{t} &= \colorunderline{Lavender}{\diff*{\left(\diffp{L}{\dot{q}^k}\right)}{t}\dot{q}^k} + \colorunderline{Red}{\diffp{L}{\dot{q}^k}\diff{\dot{q}^k}{t}} - \diffp{L}{t} - \colorunderline{Lavender}{\diffp{L}{q^k}\diff{q^k}{t}} - \colorunderline{Red}{\diffp{L}{\dot{q}^k}\diff{\dot{q}^k}{t}}\\
                    &= \left[\diff*{\left(\diffp{L}{\dot{q}^k}\right)}{t}- \diffp{L}{q^k}\right]\dot{q}^k - \diffp{L}{t}.
    \end{align*}
    Desse modo, para uma solução das equações de Euler-Lagrange, segue que \(\diff{h}{t} = -\diffp{L}{t}.\)
\end{proof}


% \begin{lemma}{Campo eletromagnético constante}{campo_constante}
%     Para um campo eletromagnético constante, isto é, \(\diffp{\vetor{E}}{t} = 0\) e \(\diffp{\vetor{B}}{t} = 0\), existe um potencial escalar \(\tilde{\phi}\) e um potencial vetor \(\vetor{\tilde{A}}\) que satisfazem \(\diffp{\tilde{\phi}}{t} = 0\) e \(\diffp{\vetor{\tilde{A}}}{t} = 0\).
% \end{lemma}
% \begin{proof}
%     Sejam \(\vetor{A}\) e \(\phi\) potenciais tais que \(\vetor{B} = \vetor{\nabla}\times\vetor{A}\) e \(\vetor{E} = -\vetor{\nabla}\phi - \diffp{\vetor{A}}{t}\). Como o campo magnético é constante, segue que \(\diffp{\vetor{A}}{t}\) é irrotacional. Portanto, como \(\mathbb{R}^3\) é simplesmente conexo, existe um campo escalar diferenciável \(\psi\) tal que \(\diffp{\vetor{A}}{t} = -\vetor{\nabla}\psi\).
%
%     Consideremos a transformação de calibre
%     \begin{equation*}
%         \vetor{\tilde{A}} = \vetor{A} + \vetor{\nabla}\psi\quad\text{e}\quad\tilde{\phi} = \phi -\diffp{\psi}{t},
%     \end{equation*}
%     de modo que \(\diffp{\vetor{\tilde{A}}}{t} = 0\) e
%     \begin{equation*}
%         \vetor{E} = -\vetor{\nabla}\tilde{\phi}.
%     \end{equation*}
%     Dessa forma, como \(\diffp{\vetor{E}}{t} = 0\), temos \(\diffp{\tilde{\phi}}{t}\) constante em relação ao tempo, isto é,
%     \begin{equation*}
%         \diffp{\tilde{\phi}}{t} = \alpha(\vetor{r}).
%     \end{equation*}
%     Ainda, como \(\vetor{\nabla}\alpha = \vetor{0}\), segue que \(\alpha\) é uma constante. Para que o potencial seja nulo no infinito, devemos ter \(\alpha = 0\), isto é
%     \begin{equation*}
%         \tilde{\phi} = \tilde{\phi}(\vetor{r}),
%     \end{equation*}
%     portanto o potencial escalar não depende do tempo.
% \end{proof}

\begin{exercício}{Energia de uma partícula em um campo eletromagnético externo}{exercício02}
    Uma partícula encontra-se na presença de um campo eletromagnético independente do tempo. Utilizando o formalismo lagrangiano, obtenha a energia do sistema.
\end{exercício}
\begin{proof}[Resolução]
    Pelo \cref{lem:lagrangiana_lorentz}, a lagrangiana de uma partícula de massa \(m\) e carga \(Q\) em um campo eletromagnético externo com potencial escalar \(\phi\) e potencial vetor \vetor{A} é dada por
    \begin{equation*}
        L = \frac12 m g_{ij} \dot{x}^i \dot{x}^j - Q\phi + Q g_{ij} A^i \dot{x}^j,
    \end{equation*}
    onde \(x^i\) são as suas coordenadas cartesianas e \(g_{ij}\) o tensor métrico Euclidiano, a função energia neste caso é dada por
    \begin{align*}
        h &= \left(mg_{ij} \dot{x}^i \delta^j_k + Q g_{ij} A^i \delta^j_k\right)\dot{x}^k - \left( \frac12 m g_{ij} \dot{x}^i \dot{x}^j - Q\phi +Q g_{ij} A^i \dot{x}^j\right)\\
          &= m g_{ik}\dot{x}^i \dot{x}^k + Qg_{ik}A^i\dot{x}^k - \left(\frac12 m g_{ik} \dot{x}^i \dot{x}^k - Q\phi + Qg_{ik} A^i \dot{x}^k\right)\\
          &= \frac12 m g_{ik} \dot{x}^i\dot{x}^k + Q\phi\\
          &= \frac12 m \inner{\vetor{v}}{\vetor{v}} + Q\phi.
    \end{align*}
    \todo[No caso em que o campo eletromagnético não depende do tempo, temos que a lagrangiana não tem dependência explícita do tempo.] Dessa forma, pelo \cref{lem:energia}, segue que \(h\) é uma quantidade conservada.
\end{proof}
