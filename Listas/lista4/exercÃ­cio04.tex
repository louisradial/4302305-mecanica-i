\begin{exercício}{Massa em uma cunha}{exercício4}
    Uma cunha de massa \(M\) repousa sobre um plano horizontal. O ângulo do plano inclinado com a horizontal é \(\alpha\). Um corpo de massa \(m\) é colocado sobre o plano inclinado com seu centro de massa a uma altura \(h\) deste. Desprezando o atrito e usando o formalismo lagrangiano:
    \begin{enumerate}[label=(\alph*)]
        \item obtenha a lagrangiana que descreve o sistema;
        \item obtenha as equações de movimento;
        \item obtenha a solução para o movimento da cunha e do corpo de massa \(m\) assumindo que no instante inicial o corpo e a cunha encontram-se parados;
        \item há alguma quantidade conservada?
    \end{enumerate}
\end{exercício}
\begin{proof}[Resolução]

\end{proof}
