\section*{Exercício 1}
\begin{lemma}{Partícula em um campo eletromagnético externo}{lagrangiana_lorentz}
    A lagrangiana de uma partícula de massa \(m\) e carga \(e\) em um campo eletromagnético externo definido pelo potencial escalar \(\phi\) e pelo potencial vetor \(\vetor{A}\) é dada por
    \begin{equation*}
        L = \frac12 m g_{ij} \dot{x}^i \dot{x}^j - e \phi + e g_{ij} A^i \dot{x}^j,
    \end{equation*}
    onde \(g_{ij}\) é o tensor métrico Euclidiano.
\end{lemma}
\begin{proof}
    Uma partícula de massa \(m\) e carga \(e\) em um campo eletromagnético externo está sujeita à força de Lorentz dada por
    \begin{equation*}
        \vetor{F} = e\left(\vetor{E} + \vetor{v}\times\vetor{B}\right),
    \end{equation*}
    onde \(\vetor{E}\) é o campo elétrico e \(\vetor{B}\) é o campo magnético.

    Podemos definir os campos elétrico e magnético a partir de um potencial escalar \(\phi\) e um potencial vetor \(\vetor{A}\),
    \begin{equation*}
        \vetor{E} = -\vetor{\nabla}\phi - \diffp{\vetor{A}}{t}\quad\text{e}\quad\vetor{B} = \vetor{\nabla} \times \vetor{A},
    \end{equation*}
    de modo que as equações de Maxwell ainda sejam satisfeitas. Neste caso, a força de Lorentz é dada por
    \begin{equation*}
        \vetor{F} = e\left(-\vetor{\nabla}\phi - \diffp{\vetor{A}}{t} + \vetor{v}\times (\vetor{\nabla}\times \vetor{A})\right).
    \end{equation*}

    Notemos que
    \begin{equation*}
        \diff{\vetor{A}}{t} = \diffp{\vetor{A}}{x^i}\dot{x}^i + \diffp{\vetor{A}}{t} \implies -\diffp{A}{t} = \inner{\vetor{v}}{\vetor{\nabla}}\vetor{A} - \diff{\vetor{A}}{t}
    \end{equation*}
    e que
    \begin{equation*}
        \vetor{v} \times (\vetor{\nabla}\times \vetor{A}) = \vetor{\nabla}\inner{\vetor{v}}{\vetor{A}} - \inner{\vetor{v}}{\vetor{\nabla}}\vetor{A},
    \end{equation*}
    uma vez que \(\diffp{\vetor{v}}{x^i} = 0\). Dessa forma, segue que
    \begin{align*}
        \vetor{F} &= e\left(-\vetor{\nabla}\phi - \diff{\vetor{A}}{t} + \vetor{\nabla}\inner{\vetor{v}}{\vetor{A}}\right)\\
                  &= e \left[- \vetor{\nabla}(\phi - \inner{\vetor{v}}{\vetor{A}}) - \diff{\vetor{A}}{t}\right].
    \end{align*}

    Definindo o operador \(\vetor{\nabla}_{\vetor{v}} = \vetor{e}_x\diffp*{}{v_x} + \vetor{e}_y\diffp*{}{v_y} + \vetor{e}_z\diffp*{}{v_z}\) e notando que tanto \(\vetor{A}\) quanto \(\phi\) não dependem das velocidades, temos
    \begin{equation*}
        -\vetor{A} = \vetor{\nabla}_{\vetor{v}}\left(\phi - \inner{\vetor{v}}{\vetor{A}}\right),
    \end{equation*}
    de modo que a força de Lorentz seja dada por
    \begin{equation*}
        \vetor{F} = e \left[- \vetor{\nabla}(\phi - \inner{\vetor{v}}{\vetor{A}}) + \diff*{\vetor{\nabla}_{\vetor{v}}\left(\phi - \inner{\vetor{v}}{\vetor{A}}\right)}{t}\right].
    \end{equation*}

    Isto é, mostramos que a força de Lorentz resulta de um potencial generalizado \(U = e\phi - e\inner{\vetor{v}}{\vetor{A}}\)
    \begin{equation*}
        F_k = - \diffp{U}{x^k} + \diff*{\left(\diffp{U}{\dot{x}^k}\right)}{t},
    \end{equation*}
    portanto a lagrangiana é dada por
    \begin{align*}
        L &= \frac12 m\inner{\vetor{v}}{\vetor{v}} - e\phi + e\inner{\vetor{v}}{\vetor{A}}\\
          &= \frac12 mg_{ij} \dot{x}^i \dot{x}^j - e\phi + eg_{ij}A^i\dot{x}^j,
    \end{align*}
    o que conclui a demonstração.
\end{proof}

\begin{exercício}{Movimento em um campo magnético uniforme}{exercício01}
    Uma partícula de massa \(m\) move-se na presença de um campo magnético constante \(\vetor{B} = B \vetor{e}_z\).
    \begin{enumerate}[label=(\alph*)]
        \item Mostre que o potencial vetor \(\vetor{A} = \frac{B}{2}(-y\vetor{e}_x + x \vetor{e}_y)\) está associado a este campo magnético.
        \item Utilizando o formalismo Lagrangiano obtenha a equação de movimento desta partícula.
        \item Obtenha a trajetória desta partícula utilizando a condição inicial que \(\vetor{r}(0) = \vetor{0}\) e que \(\vetor{v}(0) = a\vetor{e}_x + b\vetor{e}_y,\) onde \(a\) e \(b\) são constantes.
    \end{enumerate}
\end{exercício}
\begin{proof}[Resolução]
    Notemos que
    \begin{equation*}
        \vetor{\nabla} \times \left(- y\vetor{e}_x + x\vetor{e}_y\right) = 2 \vetor{e}_z,
    \end{equation*}
    portanto o potencial vetor \(\vetor{A}\) dado é associado ao campo magnético uniforme \(\vetor{B}\).

    Pelo \cref{lem:lagrangiana_lorentz}, a lagrangiana de uma partícula de carga \(Q\) e massa \(m\) em um campo eletromagnético é dada por
    \begin{equation*}
    L = \frac12 m\inner{\vetor{v}}{\vetor{v}} - Q\phi + Q \inner{\vetor{v}}{\vetor{A}},
    \end{equation*}
    onde \(\vetor{v}\) é a velocidade da partícula e \(\phi\) é o potencial escalar. Neste caso, como não há a presença de um campo elétrico, e como \(\diffp{\vetor{A}}{t} = 0\), segue que \(\phi\) é constante. Desse modo, a lagrangiana pode ser escrita como
    \begin{equation*}
        L(q^1,\dot{q}^1, q^2, \dot{q}^2) = \frac12 m g_{ij}\dot{q}^i \dot{q}^j + \frac{QB}{2} \epsilon_{ij}q^i\dot{q}^j,
    \end{equation*}
    onde \(q^1 = x\), \(q^2 = y\), \(\epsilon_{ij}\) é o símbolo de Levi-Civita, e \(g_{ij}\) é o tensor métrico Euclidiano. Assim, pelas equações de Euler-Lagrange temos
    \begin{align*}
        \diff*{\left(\diffp{L}{\dot{q}^k}\right)}{t} - \diffp{L}{q^k} = 0 &\implies \diff*{\left(\frac12 mg_{ij} \delta^i_k \dot{q}^j + \frac12 g_{ij} \dot{q}^i\delta^j_k + \frac{QB}{2}\epsilon_{ij} q^i \delta^j_k\right)}{t} - \frac{QB}{2} \epsilon_{ij}\delta^i_k \dot{q}^j = 0\\
                                                                          &\implies \diff*{\left(mg_{kj} \dot{q}^j + \frac{QB}{2}\epsilon_{ik} q^i \right)}{t} - \frac{QB}{2} \epsilon_{kj}\dot{q}^j = 0\\
                                                                          &\implies mg_{kj} \ddot{q}^j + \frac{QB}{2}\epsilon_{ik} \dot{q}^i - \frac{QB}{2} \epsilon_{kj}\dot{q}^j = 0\\
                                                                          &\implies m\ddot{q}_k + QB\epsilon_{ik} \dot{q}^i = 0.
    \end{align*}
    De forma explícita, temos as equações de movimento
    \begin{equation*}
        \begin{cases}
            \ddot{q}_1 = \omega\dot{q}_2\\
            \ddot{q}_2 = -\omega\dot{q}_1,
        \end{cases} \implies
        \begin{cases}
            \ddot{x} = \omega\dot{y}\\
            \ddot{y} = -\omega\dot{x},
        \end{cases}
    \end{equation*}
    onde \(\omega = \frac{QB}{m}\).

    Integrando a primeira equação em relação ao tempo no intervalo \([0,t]\), temos
    \begin{equation*}
        \dot{x}(t) - \dot{x}(0) = \omega(y(t) - y(0)) \implies \dot{x}(t) = a + \omega y(t).
    \end{equation*}
    Substituindo na segunda equação, temos a equação diferencial linear não homogênea
    \begin{equation*}
        \ddot{y} + \omega^2 y = - \omega a,
    \end{equation*}
    cuja solução é
    \begin{equation*}
        y(t) = \alpha \cos(\omega t) + \beta \sin(\omega t) - \frac{a}{\omega},
    \end{equation*}
    para constantes de integração \(\alpha, \beta\). Como \(y(0) = 0\) e \(\dot{y}(0) = b\), temos
    \begin{equation*}
        y(t) = \frac{b}{\omega}\sin(\omega t) -  \frac{a}{\omega}\left[1 - \cos(\omega t)\right].
    \end{equation*}
    Assim,
    \begin{equation*}
        \dot{x}(t) = b\sin(\omega t) +  a \cos(\omega t) \implies x(t) = \frac{b}{\omega}\left[1 - \cos(\omega t)\right]+\frac{a}{\omega} \sin(\omega t).
    \end{equation*}

    Deste modo, a partícula tem posição dada por
    \begin{align*}
        \vetor{r}(t) &= \vetor{r}_0 + \left[\frac{a}{\omega} \sin(\omega t) - \frac{b}{\omega}\cos(\omega t)\right]\vetor{e}_x + \left[\frac{b}{\omega} \sin(\omega t) + \frac{a}{\omega}\cos(\omega t)\right]\vetor{e}_y\\
                     &= \vetor{r}_0 + \rho\left[\cos(\varphi - \omega t)\vetor{e}_x + \sin(\varphi-\omega t)\vetor{e}_y\right],
    \end{align*}
    onde \(\vetor{r}_0 = \frac{b}{\omega}\vetor{e}_x - \frac{a}{\omega}\vetor{e}_y\), \(\rho = \sqrt{\frac{a^2 + b^2}{\omega^2}}\) e \(\varphi \in [0, 2\pi]\) é tal que
    \begin{equation*}
        \rho\cos\varphi = -\frac{b}{\omega}\quad\text{e}\quad\rho\sin\varphi = \frac{a}{\omega}.
    \end{equation*}
    Notemos que
    \begin{equation*}
        \inner{\vetor{r}(t) - \vetor{r}_0}{\vetor{r}(t) - \vetor{r}_0} = \rho^2,
    \end{equation*}
    portanto a a trajetória da partícula descreve um círculo de raio \(\rho\) centrado em \(\vetor{r}_0\) com frequência angular constante \(\omega\).
\end{proof}
