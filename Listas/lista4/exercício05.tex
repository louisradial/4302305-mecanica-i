\section*{Exercício 5}
\begin{exercício}{Massas oscilando em um eixo livre}{exercício5}
    Dois corpos idênticos de massa \(m\) podem mover-se sem atrito ao longo de uma haste e de forma simétrica como mostra a \cref{fig:teste}.
    \begin{center}
        \includegraphics[width=0.3\linewidth]{exercício05.png}
        \captionof{figure}{Sistema do \cref{ex:exercício5}}
        \label{fig:teste}
    \end{center}
    A massa da barra é desprezível e pode rodar livremente em torno do ponto \(O\). Cada massa \(m\) está conectada à origem por uma mola de constante elástica \(\frac12m\omega_0^2\).
    \begin{enumerate}[label=(\alph*)]
        \item Obtenha a lagrangiana que descreve o sistema.
        \item Obtenha as equações de movimento.
        \item Há alguma quantidade conservada? Interprete.
    \end{enumerate}
\end{exercício}
\begin{proof}[Resolução]
    Como os corpos se encontram ao longo da haste, temos
    \begin{equation*}
        \vetor{r}_1 = r_1\vetor{e}_r\quad\text{e}\quad \vetor{r}_2 = - r_2 \vetor{e}_r,
    \end{equation*}
    onde \(\vetor{e}_r = \cos\phi\sin\theta \vetor{e}_x + \sin\phi\sin\theta \vetor{e}_y + \cos\theta \vetor{e}_z\). Segue que a energia potencial do sistema é dada por
    \begin{equation*}
        U = \frac14 m \omega_0^2\left(r_1^2 + r_2^2\right) + mg\left(r_1 - r_2\right)\cos\theta
    \end{equation*}
    e a energia cinética por
    \begin{equation*}
        T = \frac12 m\left[\dot{r}_1^2 + \dot{r}_2^2 + \left(r_1^2 + r_2^2\right)\left(\dot{\theta}^2 + \dot{\phi}\sin^2\theta\right)\right].
    \end{equation*}
    Utilizando o vínculo de que o movimento ocorre de forma simétrica, \(r \equiv r_1 = r_2\), temos a lagrangiana
    \begin{equation*}
        L = m\left[\dot{r}^2 + r^2\left(\dot{\theta}^2 + \dot{\phi}^2\sin^2\theta\right)\right] - \frac12 m \omega_0^2r^2
    \end{equation*}
    para descrever o sistema.

    Pelas equações de Euler-Lagrange, temos
    \begin{equation*}
        \left\{\begin{aligned}
                \diff*{\left(\diffp{L}{\dot{r}}\right)}{t} - \diffp{L}{r} = 0 &\implies 2m\ddot{r} - 2mr\left(\dot{\theta}^2 + \dot\phi^2\sin^2\theta\right) + m\omega_0^2 r = 0\\
                \diff*{\left(\diffp{L}{\dot{\theta}}\right)}{t} - \diffp{L}{\theta} = 0 &\implies 2mr^2\ddot{\theta} + 4mr\dot{r}\dot{\theta} - 2mr^2\dot\phi^2\sin\theta\cos\theta = 0\\
                \diff*{\left(\diffp{L}{\dot{\phi}}\right)}{t} - \diffp{L}{\phi} = 0 &\implies \diff*{\left(2mr^2\dot\phi\sin^2\theta\right)}{t} = 0
        \end{aligned}\right.
    \end{equation*}
    Como a lagrangiana não depende de forma explícita do tempo e como \(\phi\) é uma coordenada cíclica temos que a energia e a componente \(\vetor{e}_z\) do momento angular do sistema,
    \begin{equation*}
        E = T + U\quad\text{e}\quad J_z = 2mr^2\dot\phi\sin^2\theta
    \end{equation*}
    são conservadas. De fato, o momento angular do sistema é dado por
    \begin{align*}
        \vetor{J} &= [\vetor{r}_1, m\dot{\vetor{r}}_1] + [\vetor{r}_2, m\dot{\vetor{r}}_2]\\
                  &= 2mr\left[\vetor{e}_r, \dot{r}\vetor{e}_r + r\dot{\theta}\vetor{e}_\theta + r\sin\theta \vetor{e}_\phi\right]\\
                  &= 2mr^2 \left(\dot\theta \vetor{e}_\phi - \dot\phi\sin\theta \vetor{e}_\theta\right),
    \end{align*}
    portanto
    \begin{align*}
        J_z &= \inner{\vetor{J}}{\vetor{e}_z}\\
            &= 2mr^2\inner{\dot\theta \vetor{e}_\phi - \dot\phi\sin\theta \vetor{e}_\theta}{\vetor{e}_z}\\
            &= 2mr^2\dot\phi\sin^2\theta,
    \end{align*}
    que é a quantidade conservada afirmada.
\end{proof}
