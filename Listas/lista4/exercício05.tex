\section*{Exercício 5}
\begin{exercício}{Massas oscilando em um eixo livre}{exercício5}
    Dois corpos idênticos de massa \(m\) podemo mover-se sem atrito ao longo de uma haste e de forma simétrica como mostra a \cref{fig:teste}.
    \begin{center}
        \includegraphics[width=0.3\linewidth]{exercício05.png}
        \captionof{figure}{Sistema do \cref{ex:exercício5}}
        \label{fig:teste}
    \end{center}
    A massa da barra é desprezível e pode rodar livremente em torno do ponto \(O\). Cada massa \(m\) está conectada à origem por uma mola de constante elástica \(\frac12m\omega_0^2\).
    \begin{enumerate}[label=(\alph*)]
        \item Obtenha a lagrangiana que descreve o sistema.
        \item Obtenha as equações de movimento.
        \item Há alguma quantidade conservada? Interprete.
    \end{enumerate}
\end{exercício}
\begin{proof}[Resolução]

\end{proof}
